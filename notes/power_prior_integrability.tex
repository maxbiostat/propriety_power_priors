\documentclass[a4paper, notitlepage, 11pt]{article}
\usepackage{geometry}
\fontfamily{times}
\geometry{verbose,tmargin=30mm,bmargin=25mm,lmargin=25mm,rmargin=25mm}
\pagestyle{empty}
% end configs
\usepackage{setspace,relsize}               
\usepackage{moreverb}                        
\usepackage{url}
\usepackage{hyperref}
\hypersetup{colorlinks=true,citecolor=blue}
\usepackage{amsmath}
\usepackage{mathtools} 
\usepackage{amsthm}
\usepackage{amssymb}
\usepackage{indentfirst}
\usepackage{todonotes}
\usepackage[authoryear,round]{natbib}
\bibliographystyle{apalike}
\usepackage[pdftex]{lscape}
\usepackage[utf8]{inputenc}

% Title Page
\title{\vspace{-9ex}\centering \bf On the propriety of power priors}
\author{
% Luiz Max de Carvalho\\
% Program for Scientific Computing (PROCC), Oswaldo Cruz Foundation. \\
}
\DeclareMathOperator*{\argmin}{arg\,min}
\DeclareMathOperator*{\argmax}{arg\,max}
\newtheorem{theorem}{Theorem}[]
\newtheorem{proposition}{Proposition}[]
\newtheorem{remark}{Remark}[]
\setcounter{theorem}{0} % assign desired value to theorem counter
\begin{document}
\maketitle

% \begin{abstract}
% 
% Key-words: ;; ; ; . 
% \end{abstract}

\section*{Background}

Power priors are cool. [bunch of citations for propriety of GLM, Cox model, exponential family, etc]

\section*{Results}

The data $x \in \mathcal{X} \subseteq \mathbb{R}^p$  and the parameter $\theta \in \Theta \subseteq \mathbb{R}^q$ and $a_0 \geq 0$.
\begin{equation}
\label{eq:power_prior}
 p(\theta | x, a_0) \propto L(x | \theta)^{a_0}\pi(\theta)
\end{equation}

We want to show under which conditions
\[\int_{\Theta} L(x | \theta)^{a_0}\pi(\theta) d\theta <\infty.\]
\begin{theorem}
\label{thm:integrability}
 Assume $\int_{\mathcal{X}} L(x | \theta)dx < \infty$.
 Denote $f_{a_0}(x;\theta) : = L(x | \theta)^{a_0}\pi(\theta)$.
 Then for $a_0 > 0$, $\int_{\Theta} f_{a_0}(x;\theta) d\theta <\infty$.
\end{theorem}
\begin{proof}
First, note that for $0 < a_0 \leq 1$, the function $g(x) = x^{a_0}$ is concave.
Then, by Jensen's inequality and the finiteness of $L(x | \theta)$ for all of its arguments we have
\[ \int_{\Theta} f_{a_0}(x; \theta) d\theta \leq \left[ \int_{\Theta} L(x|\theta)\pi(\theta)d\theta \right]^{a_0} < \infty. \]
Rewrite $f_{a_0}(x; \theta)$ as
\[ f_{a_0}(x; \theta) = L(x|\theta)^{a_0 -1} L(x|\theta)\pi(\theta). \]
If $1 \leq a_0 \leq 2$, we have the Jensen's inequality case above, since we know that $L(x|\theta)\pi(\theta)$ is normalisable.
Similarly, if $2 \leq a_0 \leq 3$, we can write 
\[  f_{a_0}(x; \theta) = L(x|\theta)^{a_0-p} L(x|\theta)^p\pi(\theta), \]
with $1 \leq p \leq 2$, again falling into the same case, since we know that $L(x|\theta)^{p}\pi(\theta)$ is normalisable.
We can then show that for any $n \geq 1 \in \mathbb{N}$, $\int_{\Theta}f_{a_0}(x; \theta)d\theta < \infty$ for  $n-1 \leq a_0 \leq n$.
The base case for $1 \leq n \leq 3$ is established.
Now suppose the hypothesis holds for $n \geq 3$.
For $n + 1$ we have for $ n \leq  a_0 \leq n + 1$ and $n-1 \leq p_n \leq n$:
\begin{align*}
 \int_{\Theta} L(x|\theta)^{a_0-p_n} L(x|\theta)^{p_n}\pi(\theta)d\theta < \infty, \\
\end{align*}
because $0 \leq a_0 - p_n \leq 1$ and $L(x|\theta)^{p_n}\pi(\theta)$ is proper by hypothesis.
\end{proof}
\begin{remark}
 The power prior on multiple historical data sets is also a proper density.
\end{remark}
\begin{proof}
 Recall that the power prior on multiple historical data sets is of the form~[Eq. 2.9]\citep{Ibrahim2015}:
 \[ \pi(\theta | \boldsymbol D, a_0) \propto \prod_{k=1}^K L(\theta | D_k)^{a_{0k}} \pi_0(\theta). \]
Assume, without loss of generality, that $L(\theta |  D_k)^{a_{0k}} > 1$ for all $\theta$ and let $m := \max(\boldsymbol a_0)$ with $\boldsymbol a_0 := \{ a_{01}, a_{02}, \ldots, a_{0K}\}$.
Then $\pi(\theta | \boldsymbol D, a_0)$ is bounded above by 
\[  g(\theta) :=  \prod_{k=1}^K L(\theta | D_k)^{m} \pi_0(\theta) =  \left[ \prod_{k=1}^K L(\theta | D_k) \right]^m  \pi_0(\theta) = L(\theta | \boldsymbol D)^m \pi_0(\theta), \]
which is normalisable following Theorem~\ref{thm:integrability}.
To relax the assumption made in the beginning, notice that this construction also bounds the case $ 0 \leq  L(\theta |  D_k)^{a_{0k}} \leq 1$ (for some $k$) above.
\end{proof}


\paragraph{TODO:}
\begin{itemize}
 \item check these proofs are valid;
 \item harmonise notation (e.g. $x$ and $D$ are the same thing, need to settle on notation);
\end{itemize}

\section{Roadmap}
The idea is to review and justify power priors for general regression models.
Since the integral of~(\ref{eq:power_prior}) with respect to $\theta$ is quite complicated, it is nice and reassuring to know that it is computable, even we don't quite know how to compute it.
So we need to
\begin{itemize}
 \item Review the necessary literature;
 \item Write down a couple examples of interesting regression models: random effects and Cox model with right-censoring (cure rate/ promotion time model);
 \item See if we can prove a few more interesting things about these models;
 \item Show our general results and argue that they provide justification for general use of power priors: they can be normalised, even if sometimes it is very hard to do so.
 \item Find and curate ``classic'' data sets and write some Stan code to fit the models above to them;
\end{itemize}
As I understand it currently, the paper would have review-y feel to it: review what has been done for regression models and then add some of new stuff we've figured out.

\section{Questions}

\begin{enumerate}
 \item Is it not the case that when employing an unormalised power prior, we can't compute marginal likelihoods? Does that matter in practice?
 \item Should we be pursuing ways of computing $c(a_0) := \int_{\boldsymbol \Theta} p(t | D, a_0)dt$ for general models? 
\end{enumerate}


% \section*{Acknowledgements}
\bibliography{power_prior}

% \begin{figure}[!ht]
% \centering
% \includegraphics[width=\textwidth, height = 15cm]{figures/}
% \caption{\textbf{}.
% }
% \label{fig:}
% \end{figure}
%%
% \begin{figure}
% \hfill
% \subfigure[Title A]{\includegraphics[width=5cm]{img1}}
% \hfill
% \subfigure[Title B]{\includegraphics[width=5cm]{img2}}
% \hfill
% \caption{\textbf{}.
% }
% \label{fig:}
% \end{figure}
\end{document}          
